\documentclass[final]{beamer}
\usepackage[orientation=portrait, size=custom, width=60.96, height=91.44]{beamerposter}
\usepackage{tikz}
\usetikzlibrary{arrows, shapes, backgrounds, chains, decorations, calc, fit, shadows}
\usepackage{pgf}
\usepackage{tikzuml-v1.0b/tikz-uml}
\usepackage{scalefnt}
\mode<presentation>{\usetheme{Radio}}


\title{Bidirectional Git Objects Transferring\\ by Subdirectories Specified}
\author{Congbin Guo \& Eric Wang}
\institute{VMware Inc.}

%\usetheme{Desert}
\begin{document}
\begin{frame}[t]
  \begin{columns}[t]
    \begin{column}{.5\linewidth}
      \begin{block}{Abstract}
        \input abstract.input
      \end{block}
      \begin{block}{Problems}
To support subdirectories transferring between server and client needs to
resolve below problems:
\begin{itemize}
  \item How does the client discovery the subdirectories, and negotiate with
      the server that which of them to be transferred?

  \item How does the server optimize the Git objects pack according to chosen
      subdirectories?

  \item How does the client operate with the partial Git objects store, and
      furthermore ensure the newly created commits are as good as the ones
      from fully Git objects store?
\end{itemize}
      \end{block}
      \begin{block}{Git Transfer Protocol}
        \begin{columns}[T]
          \begin{column}{.5\textwidth}
            Current Git transfer protocol:
\begin{enumerate}
  \item Client initiates the request by calling \verb|git-upload-pack| on
    server.

  \item Server advertises all current reference SHA1s ($H_s$) to the client.

  \item Client sends a list of SHA1 of \verb|want| ($H_w$) and \verb|have|
    ($H_h$), which indicates the start and end SHA1 of commit objects to be
    transferred.
    In case of cloning, the list of \verb|have| is skipped since the client has
    nothing yet.

  \item Server calculates all commit objects (and all other Git objects
    associated) in range of $[H_w, H_h)$, then writes to a stream connected to
    the client.
\end{enumerate}

Extend Git transfer protocol primitives:

\begin{itemize}
  \item \verb|dir| get a list of files and directories under \verb|path| of a
    \verb|SHA1|, which equivalent to what command
    \verb|git ls-tree SHA1 -- path| does.

  \item \verb|include| specify files or subdirectories to be included in data
    transferring.

  \item \verb|exclude| specify files or subdirectories to be excluded from data
    transferring.
\end{itemize}
          \end{column}
          \begin{column}{.5\textwidth}
            \begin{figure} \centering
              \resizebox{1\textwidth}{!}{
                \includegraphics{figures/git-proto-clone-seq.pdf}
              }
              \caption{Current Git Transfer Protocol}
            \end{figure}
            \begin{figure} \centering
              \resizebox{1\textwidth}{!}{
                \includegraphics{figures/git-proto-ext-seq.pdf}
              }
              \caption{Extended Git transfer protocol}
            \end{figure}
          \end{column}
        \end{columns}
\begin{figure}[htpb]
  \centering
  \begin{verbatim}
  upload-request = dir-commands
                   want-list
                   have-list
                   dir-requests
                   compute-end

  dir-commands = *dir-command

  dir-command = PKT-LINE(dir SP obj-id SP path LF)

  dir-requests = *(dir-request)

  dir-request = include-line | exclude-line

  include-line = PKT-LINE(include SP path LF)

  exclude-line = PKT-LINE(exclude SP path LF)
  \end{verbatim}
  \caption{ABNF of extended Git transfer protocol}
  \label{fig:git-proto-ext-ABNF}
\end{figure}

      \end{block}
    \end{column}
    \begin{column}{.5\linewidth}
      \begin{block}{Git Objects Packing Optimization}
        \begin{columns}[T]
          \begin{column}{.5\textwidth}
\begin{itemize}
  \item We keep tag objects because they have minor impact to repository size.

  \item We keep commit objects because they are the backbone of the repository
    and ensure the consistency.

  \item We remove blob and tree objects in case the files or directories they
    represent are not in the subdirectories specified.

\end{itemize}
          \end{column}
          \begin{column}{.5\textwidth}
              \begin{figure} \centering
                \resizebox{1\textwidth}{!}{
                  \includegraphics{figures/git-model-base.pdf}
                }
              \end{figure}
              \begin{figure} \centering
                \resizebox{1\textwidth}{!}{
                  \includegraphics{figures/git-model-filtering.pdf}
                }
              \end{figure}
              \begin{figure} \centering
                \resizebox{1\textwidth}{!}{
                  \includegraphics{figures/git-model-final.pdf}
                }
              \end{figure}
          \end{column}
        \end{columns}
      \end{block}
      \begin{block}{System Impelementation}
        \begin{columns}[T]
          \begin{column}{.3\textwidth}
\begin{itemize}
  \item \emph{Subdirs} communicates with client to track which subdirectories
    of the repository to be transferred.

  \item \emph{Filter} gets the objects list from
    \verb|git rev-list| command, and filters them by subdirectories specified.
    The component has two outputs.
    One is \emph{real object list} that passes to object-packing component.
    Another is \emph{mock object list}, which is a subset of filtered out
    objects, passes to \emph{mock object builder} on client.

  \item \emph{Mock object builder} accepts the mock object list
    from the filter, and then create mock objects accordingly.
    It builds the mock objects in pack file format\cite{packformat} for better
    performance.

  \item \emph{Commit creator} creates new commits and ensures
    the tree object of commit is complete as described in Section
    \ref{sec:create-new-commit}.
\end{itemize}
          \end{column}
          \begin{column}{.7\textwidth}
              \begin{figure} \centering
                \resizebox{1\textwidth}{!}{
                  \includegraphics{figures/components.pdf}
                }
              \end{figure}
          \end{column}
        \end{columns}
      \end{block}
    \end{column}
  \end{columns}
\end{frame}
\end{document}
