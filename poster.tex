\documentclass[final]{beamer}
\setbeamertemplate{caption}[numbered]
\usepackage[orientation=portrait, size=custom, width=60.96, height=91.44]{beamerposter}
\usepackage{tikz}
\usetikzlibrary{arrows, shapes, backgrounds, chains, decorations, calc, fit, shadows}
\usepackage{pgf}
\usepackage{tikzuml-v1.0b/tikz-uml}
\usepackage{scalefnt}
\usepackage{wrapfig}
\mode<presentation>{\usetheme{Radio}}


\title{Bidirectional Git Objects Transferring\\ by Subdirectories Specified}
\author{Congbin Guo \& Eric Wang \{cguo,wange\}@vmware.com}
\institute{VMware Inc.}

%\usetheme{Desert}
\begin{document}
\begin{frame}[t, fragile]
  \begin{columns}[t]
    \begin{column}{.5\linewidth}
      \begin{block}{Abstract}
        \input abstract.input
      \end{block}
      \begin{block}{Problems}
\begin{itemize}
    \justifying
  \item How does the client discover the subdirectories, and negotiate with
      the server that which of them to be transferred?

  \item How does the server optimize the Git objects pack according to chosen
      subdirectories?

  \item How does the client operate with the partial Git objects store, and
      furthermore ensure the newly created commits are as good as the ones
      from fully Git objects store?
\end{itemize}
      \end{block}
      \begin{block}{Git Object Model \& Transfer Protocol}
        \begin{columns}[T]
          \begin{column}{.5\textwidth}
              \begin{figure}\centering
                \resizebox{\textwidth}{!}{
                  \includegraphics{figures/workspace}
                }
                \caption{Snapshots of the example workspace}
                \label{fig:workspace}
              \end{figure}
            Current Git transfer protocol:
\begin{enumerate}
    \justifying
  \item Client initiates the request by calling \verb|git-upload-pack| on
    server.

  \item Server advertises all current reference SHA1s ($H_s$) to the client.

  \item Client sends a list of SHA1 of \verb|want| ($H_w$) and \verb|have|
    ($H_h$), which indicates the start and end SHA1 of commit objects to be
    transferred.
    In case of cloning, the list of \verb|have| is skipped since the client has
    nothing yet.

  \item Server calculates all commit objects (and all other Git objects
    associated) in range of $[H_w, H_h)$, then writes to a stream connected to
    the client.
\end{enumerate}
          \end{column}
          \begin{column}{.5\textwidth}
              \begin{figure}\centering
                \resizebox{\textwidth}{!}{
                  \includegraphics{figures/git-model-base}
                }
                \caption{Git repository for an example workspace}
                \label{fig:git-repo}
              \end{figure}
            \begin{figure} \centering
              \resizebox{\textwidth}{!}{
                \includegraphics{figures/git-proto-clone-seq.pdf}
              }
              \caption{Current Git Transfer Protocol}
            \end{figure}
          \end{column}
        \end{columns}
      \end{block}
      \begin{block}{Extended Git Transfer Protocol}

        \begin{columns}[T]
          \begin{column}{.5\textwidth}
\begin{itemize}
    \justifying
  \item Subdirectories Negotiation
    \begin{enumerate}
  \item \verb|dir| get a list of files and directories under \verb|path| of a
    \verb|SHA1| (equivalent to \verb|git ls-tree SHA1 -- path|).

  \item \verb|include| specify files or subdirectories to be included in data
    transferring.

  \item \verb|exclude| specify files or subdirectories to be excluded from data
    transferring.
      \end{enumerate}
\end{itemize}
\end{column}
        %\begin{wrapfigure}{r}{.5\textwidth}
          %\begin{minipage}{.5\textwidth}
          \begin{column}{.5\textwidth}
            \begin{figure} \centering
              \resizebox{\textwidth}{!}{
                \includegraphics{figures/git-proto-ext-seq.pdf}
              }
              \caption{Extended Git transfer protocol}
            \end{figure}
          %\end{minipage}
        %\end{wrapfigure}
          \end{column}
        \end{columns}

      \end{block}
    \end{column}
    \begin{column}{.5\linewidth}
      \begin{block}{Git Objects Packing Optimization}
        \begin{columns}[T]
            \begin{column}{.5\textwidth}
\begin{itemize}
    \justifying
  \item We remove some objects to reduce the amount of objects to transfer:
    \begin{enumerate}
  \item We keep tag objects because they have minor impact to repository size.

  \item We keep commit objects because they are the backbone of the repository
    and ensure the consistency.

  \item We remove blob and tree objects in case the files or directories they
    represent are not in the subdirectories specified.
\end{enumerate}


\item \emph{Git Mock Object} a client side Git object file with \textbf{EMPTY}
  content.

\item Command line to create Git mock objects:
\begin{verbatim}
  # create an empty blob object
  $ :> empty_file
  $ git hash-object -w empty_file
  e69de29...48c5391

  # create an empty tree object
  $ git mktree <<EOF
  EOF
  4b825dc...bee4904
\end{verbatim}
\end{itemize}
\end{column}
          \begin{column}{.5\textwidth}
              \begin{figure} \centering
                \resizebox{1\textwidth}{!}{
                  \includegraphics{figures/git-model-filtering.pdf}
                }
                \caption{Objects can be excluded in the example repository}
                \label{fig:find-obj-to-remove}
              \end{figure}
              \begin{figure} \centering
                \resizebox{1\textwidth}{!}{
                  \includegraphics{figures/git-model-final.pdf}
                }
                \caption{Replacing Git objects with mock objects}
                \label{fig:mock-objects}
              \end{figure}
            \end{column}
\end{columns}
      \end{block}
      \begin{block}{System Impelementation}
            %\begin{wrapfigure}{r}{.7\textwidth}
              %\begin{minipage}{.7\textwidth}
              \begin{figure} \centering
                \resizebox{.85\textwidth}{!}{
                  \includegraphics{figures/components.pdf}
                }
                \caption{System level components}
              \end{figure}
            %\end{minipage}
          %\end{wrapfigure}
\begin{itemize}
    \justifying
  \item \emph{Subdirs} communicates with client to track which subdirectories
    of the repository to be transferred.

  \item \emph{Filter} gets the objects list from
    \verb|git rev-list| command, and filters them by subdirectories specified.

  \item \emph{Mock object builder} accepts the mock object list
    from the filter, and then create mock objects accordingly.

  \item \emph{Commit creator} creates new commits and ensures
    the tree object of commit is complete as described.
\end{itemize}
      \end{block}
    \end{column}
  \end{columns}
\end{frame}
\end{document}
