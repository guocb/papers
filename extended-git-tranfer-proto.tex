\documentclass[preprint]{sigplanconf}

\usepackage[pdftex]{graphicx}
\usepackage[all]{xy}
\usepackage{tikz}
\usetikzlibrary{arrows, shapes, backgrounds, chains, decorations, calc, fit,
    shadows}
\usepackage[english]{babel}
\usepackage[Q=yes]{examplep}
\usepackage{pgf}
\usepackage{listings}

\lstset{
  basicstyle=\footnotesize\ttfamily
}

\begin{document}
\title{Extended Git Transfer Protocol by Subdirectories of A Repository}
\authorinfo{Congbin Guo\and Eric Wang}{SCM team, VMware Inc.}{\{cguo,wange\}@vmware.com}
\maketitle
\abstract{
  % the problem
  The lack of partial cloning/fetching/pushing by subdirectories consumes lots of time and network bandwidth when talks to a remote Git repository.
  % why interesting
  This downgrades the development efficiency against many large enterprise level repositories.
  % solution
  This paper extends the current Git transfer protocol to support bidirectional transferring by subdirectories of a repository specified by user.
  % value
  The presented extended protocol improves the development efficiency and keeps the Git experience as good as common Git work flows, finally help the Git adoption in enterprises.
}

\keywords
Git transfer protocol, large repository

\section{Introduction}
%(
% describe the problem

% my contribution
- extend teh git transferring protocol
  - extend the transferring protocol to enable it show the directories of the repo
  - describe the steps to select specific subdirectories
  - describe how server side handle the directories and create proper pack files
  - describe how client side create mock blob and tree objects 
  - describe how client create proper commit object based on mocked objects.
- show other technologies which can be combined to use 

%)
\section{The problem}
%(

%)

\section{My idea}

\section{The details}

\section{Related work}

\section{Conclusions and Future works}


\end{document}
